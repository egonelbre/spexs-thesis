\chapter{SPEXS}

SPEXS is an pattern discovery algorithm described by Vilo et al.
This algorithm finds patterns from a sequence.
We take this as our basis for developing a new parallel algorithm.
In this chapter we describe original algorithm so that we can
later show the changes made to this algorithm.

\section{Algorithm}

We describe the general representation of the SPEXS algorithm. 
The original algorithm was as follows:

\begin{verbatim}
TODO: format nicely
Algorithm 3.19 The SPEXS algorithm
Input: StringS, pattern class P, output criteria, search order, and fitness measureF
Output: Patterns \pi from pattern class P fulfilling all the criteria, and output in the order
of fitnessF
Method:
1. Convert input sequences into a single sequence, initiate the data structures
2.Root  new node
3.Root:label = \eps
4.Root:pos  (1; 2; : : : ; n) // Assume empty pattern to match everywhere
5.enqueue(Q; Root; order)
6. whileN  dequeue(Q)
7. Create all possible extensionsP 2 P ofN usingN :pos andS
8. foreach extensionP ofN
9. if patternP and position listP:pos fulfill the criteria
10. then
11.N :child  P
12. calculateF(P; S)
13.enqueue(Q; P; order) // Insert toQ for further extensions
14. ifP fulfills the output criteria storeP into output queueO
15. end
16. Report the list of top-ranking patterns from output queueO
\end{verbatim}

The main idea of the algorithm is that first we generate a 
pattern and a query that match all possible positions in 
the sequence. We then put this query into a queue for extending.

Extending a query means finding all querys whose patterns length
is longer by 1. If any of the querys is fit by some criteria
it will be put into the main queue, for further extension, 
and output queue for possible result.

\chapter{Conclusions}

\todo[inline]{results 1}

\todo[inline]{results 2}

\todo[inline]{context, compare}

\todo[inline]{strength + limitations}

\todo[inline]{so what? why is it important}

\todo[inline]{what is next?}

\todo[inline]{strong conclusions}

\todo[inline]{take home message}

In this thesis we showed how by making an algorithm more abstract and
general we can also make it parallel. We showed that this algorithm
can find patterns from sequences and also NFAs.

Although we showed that we can apply this algorithm on NFAs we did
not analyze the performance characteristics. This suggests that this 
algorithm may be able to work on trees and graph, but would 
require slight modifications.

We also demonstrated how to make the algorithm more concrete 
and work well on some particular datasets. This also took into 
consideration further development and flexibility of the algorithm.

The SPEXS2 implementation currently is already used in biosequenceing
and text mining.

\chapter{Conclusions}
\label{c:conclusions}

\WIP

\tow{results 1}

\tow{results 2}

\tow{context, compare}

\tow{strength + limitations}

\tow{speed}

\tow{so what? why is it important}

\tow{what is next?}

\tow{strong conclusions}

\tow{take home message}

In this thesis we showed how by making an algorithm more abstract and general we can also make it parallel. We showed that this algorithm can find patterns from sequences.

Although we showed that we can apply this algorithm on NFAs we did not analyze the performance characteristics. This suggests that this algorithm may be able to work on trees and graph, but would require slight modifications.

We also demonstrated how to make the algorithm more concrete and work well on some particular datasets. This also took into consideration further development and flexibility of the algorithm.

The SPEXS2 implementation currently is already used in biosequenceing and text mining.

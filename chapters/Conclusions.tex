\chapter{Conclusions}
\label{c:conclusions}

In this thesis we analyzed how to develop a parallel pattern discovery algorithm. We showed how we can take an already existing algorithm and parallelize it by generalizing, decomposing and then reifying. Finding the general idea of the algorithm can simplify the algorithm and provide more intuitive ways of interpreting it. Decomposing the algorithm allows us to talk about separate parts of the algorithm and modify them without affecting the general idea of the algorithm. If we have an abstract algorithm we can substitute those parts with parallel structures and algorithms.

As a practical part we implemented a parallelized algorithm based on \emph{spexs}\cite{spexs}. We discussed several problems of implementing an algorithm and interesting approaches to these problems. The program has been designed to be easily extendable for different inputs, filters and interestingness criteria. We discussed different possible uses for the implementation and analyzed the performance gained from parallelization.

Approaches suggested in this thesis could be used to generalize and parallelize other algorithms. Finding generic algorithms can be an easy way of discovering new optimizations, new algorithms and new potential applications for algorithms. If these generalizations can be implemented practically, we make the implementation easily extendable and also usable for a wider range of problems.

\chapter*{Paralleelne Mustriotsing}
\label{c:kokkuvote}
\addcontentsline{toc}{chapter}{\numberline{}Kokkuvõte}

\begin{flushleft}
  {\large Egon Elbre } \\[2mm]
  {\large Magistritöö } \\[6mm]
\end{flushleft}

Selles töös uurisime, kuidas arendada paralleelset mustrituvastus algoritmi. Näitasime, kuidas võtta olemasolev algoritm ning paralleliseerida see üldistades, liigendades ja reifitseerides. Algoritmi üldistamine võib tuua esile intuitiivse algoritmi interpretatsiooni. Liigendatud algoritmis on võimalik iga osa eraldi käsitleda algoritmi tulemust muutmata. Asendades iga osa paralleelsete struktuuride ja algoritmidega, saamegi paralleelse algoritmi.

Praktilise osana implementeerisime paralleliseeritud algoritmi \emph{spexs2} võttes aluseks algoritmi SPEXS\cite{spexs}. Seejärel arutlesime erinevate probleemide üle, mis tekkisid algoritmi implementeerimisel. \emph{spexs2}-te on võimalik laiendada erinevate sisendandmetega, otsingufiltritega ja huvitavuskriteeriumitega. Pakkusime välja erinevaid algoritmi kasutusvõimalusi ning analüüsisime paralleliseerimise tulemusel saavutatud kiirusevõitu.

Selles töös tutvustatud ideid saab kasutada algoritmide üldistamisel ja paralleliseerimisel. Algoritmi üldistamisel on võimalik leida uusi viise kuidas algoritmi optimeerida ning avastada uusi algoritme ja leida uusi kasutusvaldkondi sellele algoritmile. Üldistuste implementeerimisel saame programmi, mida on lihtne laiendada ning mida saab kasutada erinevate probleemide lahendamiseks.


\chapter{Conclusions}
\label{c:conclusions}

In this thesis we analyzed how to develop a parallel pattern discovery algorithm. We showed how we can take an already existing algorithm and parallelize it by generalizing, decomposing and then reifying. Finding the general idea of the algorithm can simplify the algorithm and provide more intuitive way of interpreting it. Decomposing allows us to talk about separate parts of algorithm and modify them without affecting the general idea of the algorithm. If we have an abstract algorithm we can substitute those parts with parallel structures and algorithms.

As a practical part we implemented a parallelized algorithm based on \emph{spexs}\cite{spexs}. We discussed several problems of implementing an algorithm and discussed interesting approaches to these problems. The program has been designed to be easily extendable for different inputs, filters and interestingness criteria. We discussed different possible uses for the implementation and analyzed the performance gained from parallelization.

Further work on \emph{spexs2} should investigate the best configurations for different applications. We discussed that \emph{spexs2} could be made to find tree and graph patterns, it could be an interesting research topic.

Approaches suggested in this thesis could be used to generalize and parallelize other algorithms. Finding generic algorithms can be an easy way of discovering new optimizations, new algorithms and new potential applications for algorithms. If these generalizations can be implemented practically we make the implementation easily extendable and also usable for wider range of problems.
\chapter{Algorithms}

There are many itemset discovery algorithms, but only few are general and 
can discover more complex patterns. \insertref{???}

\section{Common algorithms}

\subsection{SPEXS}

SPEXS is an pattern discovery algorithm described by Vilo et al. \cite{spexs}
This algorithm finds patterns from a sequence. We take this as our basis for developing a new parallel algorithm. In this chapter we describe original algorithm so that we can later show the changes made to this algorithm.

We describe the general representation of the SPEXS algorithm. 
The original algorithm was as follows:

\todo[inline]{move algorithm to generalization}

\begin{algorithm}[H]
	\caption{The SPEXS algorithm}
\begin{algorithmic}[1]
	\Require{String $S$, pattern class $\sym{P}$, output criteria, search order, and fitness measure $\sym{F}$}
	\Ensure{Patterns $\pi \in \sym{P}$ fulfilling all criteria, and output in the order of fitness $\sym{F}$}

	\State{Convert input sequences into a single sequence}
	\State{Initiate data structures}

	\Statef{Root \gets new node}
	\Statef{Root.label \gets \epsilon}
	\Statef{Root.pos \gets (1,2,...,n)}
	\Statef{enqueue(\sym{Q}, Root, order)}

	\While{$N \gets dequeue(\sym{Q})$}
		\State{ Create all possible 
			extensions $p \in \sym{P}$ of $N$ using $N.pos$ and $S$ }
		\For{ extension $p$ of $N$}
			\If{pattern $p$ and position list $p.pos$ fulfill the criteria}
				\Statef{N.child \gets p}
				\State{calculate $\sym{F}(p,S)$}
				\Statef{enqueue(\sym{Q},p,order)}
				\If{ $p$ fulfills the output criteria}
					\State{store $p$ into ouput queue $\sym{O}$}
				\EndIf
			\EndIf
		\EndFor
	\EndWhile
	\State{Report the list of top-ranking patterns from output queue $\sym{O}$}
\end{algorithmic}
\end{algorithm}

The main idea of the algorithm is that first we generate a 
pattern and a query that matches all possible positions in 
the sequence. We then put this query into a queue for extending.
Extending a query means finding all queries whose patterns length
is longer by 1. If any of the queries is fit, by some criteria,
it will be put into the main queue, for further extension, 
and output queue for possible output.

\subsection{TEIRESIAS}

\tow{overview} \cite{TEIRESIAS}

\subsection{Verbumculus}

\tow{overview} \cite{Verbumculus}

\subsection{MobyDick}

\tow{overview} \cite{MobyDick}

\subsection{RSAT}

\tow{overview} \cite{RSAT}

\subsection{VanHelden}

\tow{overview} \cite{VanHelden}

\section{Reviews}

\tow{about some reviews}

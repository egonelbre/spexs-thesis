\chapter{Algorithms}

There are many itemset discovery algorithms, but only few are general and can discover more complex patterns.

\section{SPEXS}

SPEXS is an pattern discovery algorithm described by Vilo et al.
This algorithm finds patterns from a sequence.
We take this as our basis for developing a new parallel algorithm.
In this chapter we describe original algorithm so that we can
later show the changes made to this algorithm.

We describe the general representation of the SPEXS algorithm. 
The original algorithm was as follows:


\begin{algorithm}[H]
	\caption{The SPEXS algorithm}
\begin{algorithmic}[1]
	\Require{String $S$, pattern class $\sym{P}$, output criteria, search order, and fitness measure $\sym{F}$}
	\Ensure{Patterns $\pi \in \sym{P}$ fulfilling all criteria, and output in the order of fitness $\sym{F}$}

	\State{Convert input into sequences into a single sequence}
	\State{Initiate data structures}

	\Statef{Root \gets new node}
	\Statef{Root.label \gets \epsilon}
	\Statef{Root.pos \gets (1,2,...,n)}
	\Statef{enqueue(\sym{Q}, Root, order)}

	\While{$N \gets dequeue(\sym{Q})$}
		\State{ Create all possible 
			extensions $P \in \sym{P}$ of $N$ using $N.pos$ and $S$ }
		\For{ extension $P$ of $N$}
			\If{pattern $P$ and position list $P.pos$ fulfill the criteria}
				\Statef{N.child \gets P}
				\State{calculate $\sym{F}(P,S)$}
				\Statef{enqueue(\sym{Q},P,order)}
				\If{ $P$ fulfills the output criteria}
					\State{store $P$ into ouput queue $\sym{O}$}
				\EndIf
			\EndIf
		\EndFor
	\EndWhile
	\State{Report the list of top-ranking patterns from output queue $\sym{O}$}
\end{algorithmic}
\end{algorithm}

The main idea of the algorithm is that first we generate a 
pattern and a query that match all possible positions in 
the sequence. We then put this query into a queue for extending.

Extending a query means finding all queries whose patterns length
is longer by 1. If any of the queries is fit by some criteria
it will be put into the main queue, for further extension, 
and output queue for possible result.


\section{TEIRESIAS}

\todo[inline]{write about it}

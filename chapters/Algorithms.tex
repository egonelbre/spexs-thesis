\chapter{Approaches to Pattern Discovery}
\label{c:algorithms}

There have been suggested several ways of organizing patten discovery algorithms\cite{SurveyDNAMotif, SurveyMotifDiscovery, CombinatorialSubtle, Hausler05}. Here we give a overview of the different ideas; for thorough descriptions we suggest "Motif Discovery on Promotor Sequences" by M. Häußler and J. Nicolas\cite{Hausler05} and "A survey of motif discovery methods in an integrated framework" by Sandve and Drabløs\cite{SurveyMotifDiscovery}.

\section{Pattern rating}

There are many ways of comparing patterns to find the "most interesting" pattern. Of course, pairwise comparison is often wasteful and it is better to have a interestingness measure. For practical purposes it is useful to represent that measure with a floating point value.

One useful property an interestingness measure can have is monotonicity when patterns are ordered by length. This means that when we move from a small pattern to a larger pattern the interestingness always either grows or decreases. For example, if we make a pattern more specific the number of matches can only decrease. This can be very helpful for pruning the search space. Mostly monotonic functions could provide probabilistic pruning, but this area doesn't have much research.

\begin{exmp}
Lets assume we are looking patterns that should have at least 10 matches. If we encounter a pattern \R{ACT} that has 8 matches then we do not have to examine patterns \R{xACT} and \R{ACTx}, where \R{x} is some token from the alphabet.
\end{exmp}

The most trivial measure for patterns is the number of pattern matches in the dataset. When calculating number of matches we must be aware that we can have multiple matches per sequence. For example, a pattern \R{AA} matches in sequence \R{AAAAAA} 6 times. It is better to count the number of sequences that contain the matches to account for such pathological cases.

Many patterns can occur by chance and removing such false positives from output is important. As previously mentioned a frequent pattern is not necessarily interesting. We can use a reference dataset to compare the frequencies of patterns. There are different possibilities for a reference dataset: background sequence, shuffled input sequences, background Markov-Model, binomial/multinomial models. The background sequence is usually a similar dataset to the dataset we are analyzing; for example, if we select a subset from data for analyzing we can use the rest as a background sequence. If there are no background sequences we can simply shuffle the input sequences to get a "randomized" sample. To preserve more of the data characteristic we can build more complex models for randomization such as hidden Markov Model\cite{RatingMarkovModel} or binomial and multinomial models.

Many rating functions are specified using the reference dataset as a measurement for false positives. One trivial way is to use the ratio between the input sequence occurrence and background sequence occurrences. One problem with ratios is that, if the frequencies are small then the ratios may be very high. Using binomial or hypergeometric model we can estimate how probable is the number of occurrences in input and background dataset. There are also measures Z-score, which estimates how many standard deviations an observation differs from the mean, and $\chi^2$-Value, which estimates whether a frequency distribution differs from theoretical distribution.

If we have several patterns with the same "score" then we can break the "tie" by comparing more measures. One such measure is complexity of a pattern. For example, when we have patterns \R{AATTGGG} and \R{C} and both have the same number of occurrences then the more specific pattern is probably more interesting.

\section{Algorithmic techniques}

\WIP

\subsection{Counting sequences}

To find the most frequent sequences the simplest approach is to enumerate all possible sequences with the number of occurrences. A trie or an hash-table can be used for effective storage and lookup for occurrences. We can use the frequency table to prune sequences that are infrequent.







word counting

combinatorial

sampling

online/offline

iterative

division

alignment

% ==================================================================================
% ==================================================================================
% ==================================================================================
% ==================================================================================
% ==================================================================================
% ==================================================================================


\subsection{Combinatorial and Statistical}

\subsection{Iterative and Division}

\subsection{Online and Offline}

The algorithms can be online or offline. Online algorithms take a stream input and process it piece-by-piece without having the entire input available from the start. Offline algorithm can examine the whole data immediately as needed.

Online algorithms respond better to change and can incrementally adjust to new data. This is especially useful when we are working with real-time data, such as network monitoring and financial data analysis.

Offline algorithms can find more complicated patterns and provide more "interesting" results. Of course more complicated patterns require more resources to find.

Pattern discovery algorithms in 


In this chapter we give a overview of different algorithms used for pattern discovery.

\hmm{Reorganize somehow}

\section{Algorithms}

Overview of different combinatorial algorithms.

\section{SPEXS}

SPEXS is an pattern discovery algorithm described in "Pattern Discovery from Biosequences"\cite{spexs}. This algorithm finds patterns from a sequence. We take this as our basis for developing a new parallel algorithm. In this chapter we describe original algorithm so that we can later show the changes made to this algorithm.

We describe the general representation of the SPEXS algorithm. 
The original algorithm was as follows:

\todo[inline]{move algorithm to generalization}

\begin{algorithm}[H]
	\caption{The SPEXS algorithm}
\begin{algorithmic}[1]
	\Require{String $S$, pattern class $\sym{P}$, output criteria, search order, and fitness measure $\sym{F}$}
	\Ensure{Patterns $\pi \in \sym{P}$ fulfilling all criteria, and output in the order of fitness $\sym{F}$}

	\State Convert input sequences into a single sequence
	\State Initiate data structures

	\Let{Root}{new node}
	\Let{Root.label}{$\epsilon$}
	\Let{Root.pos}{(1,2,...,n)}
	\State enqueue($\sym{Q}$, Root, order)

	\While{$N \gets$ dequeue($\sym{Q}$)}
		\State Create all possible extensions $p \in \sym{P}$ of $N$ using $N$.pos and $S$
		\For{ extension $p$ of $N$}
			\If{pattern $p$ and position list $p$.pos fulfill the criteria}
				\Let{$N$.child}{$p$}
				\State calculate $\sym{F}(p,S)$
				\State enqueue($\sym{Q}$,$p$,order)
				\If{$p$ fulfills the output criteria}
					\State store $p$ in output queue $\sym{O}$
				\EndIf
			\EndIf
		\EndFor
	\EndWhile
	\State{Report the list of top-ranking patterns from output queue $\sym{O}$}
\end{algorithmic}
\end{algorithm}

The main idea of the algorithm is that first we generate a 
pattern and a query that matches all possible positions in 
the sequence. We then put this query into a queue for extending.
Extending a query means finding all queries whose patterns length
is longer by 1. If any of the queries is fit, by some criteria,
it will be put into the main queue, for further extension, 
and output queue for possible output.

\subsection{TEIRESIAS}

TEIRESIAS\cite{TEIRESIAS} is an algorithm for the discovery of rigid patterns in biological sequences. \tow{more}

TEIRESIAS operates in two phases: scanning and convolution. Scanning phase identifies elementary patterns that are frequent. During convolution these elementary patterns are combined to make larger patterns.

This method is a divide and conquer method to only consider frequent patterns.

patterns with any symbol 

\tow{more} 

\subsection{Verbumculus}

Verbumculus\cite{Verbumculus} is...

statistical analysis, pattern matching

no complex patterns

\tow{more}

\subsection{MobyDick}

MobyDick\cite{MobyDick}...
statistical prediction of frequent

no complex patterns

\tow{more} 

\subsection{RSAT}

RSAT\cite{RSAT} is ...

matrix based pattern

\tow{more}

\subsection{Other}

\cite{NetworkMotifsDiscovery, GenericMotifSequential}

\section{Reviews}

\cite{CombinatorialSubtle, SurveyDNAMotif, SurveyMotifDiscovery}

\tow{about some reviews}

\section{Problems}

\WIP

Algorithms are fixed and hard to extend with new pattern types, structures and optimizations. Generalization usually comes at the cost of performance and complexity. \tow{more}

Sequential algorithms do not take advantage of multicore processors. \tow{more}

Data that exceeds computer memory can't work efficiently... can't be distributed efficiently. \tow{more}
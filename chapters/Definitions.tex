\chapter{Definitions}
\label{c:definitions}

Pattern discovery is a research area aiming to discover unknown patterns in a given set of data structures that are frequent and interesting according to some measure. In this chapter we formally define necessary terms used in this thesis.

\section{Sequence and Dataset}

We use $\Sigma$ to denote the set of tokens in the dataset, an \emph{alphabet}. 
The \emph{size} of the alphabet is $|\Sigma|$. \emph{Tokens} can be numbers, 
letters, words or sentences - any symbol.

Any sequence $S=a_1 a_2 ... a_n, \forall a_i \in \Sigma$ is called a \emph{sequence} 
over the token set $\Sigma$. If the length of the
string is $0$, it is called an empty sequence or $\epsilon$.

\begin{exmp}
\R{ACGTGCCATC} is a sequence where $\Sigma = \{\R{A}, \R{C}, \R{G}, \R{T}\}$.
\end{exmp}

A \emph{dataset} is a collection of sequences.

\begin{exmp}
In a document sentences can be considered as a \emph{dataset}, where a single sentence is a \emph{sequences} and each word is a \emph{token} in the alphabet. Text \R{This is some example. This is an other example.} has sequences \{ [\R{This is an example}], [\R{This is an other example}] \} and the alphabet is $\Sigma =$ \{\R{this}, \R{is}, \R{an}, \R{example}, \R{other}\}.	
\end{exmp}

\section{Pattern}

Our aim is to discover repetetive and common structures in data. We call such structures \emph{patterns}. A generic way to define a \emph{pattern} is as a set of all the sub-structures it represents. This means we can say whether some data sub-structure is represented by a pattern.

The \emph{pattern structure} is usually dependent on the data-structures which it represents. For example sequence patterns are usually represented sequences, graph patterns are represented as graphs; but sequence patterns could also be represented as a graph.

We denote the set of structures that a pattern structure $p$ defines as $all(p)$. If $\alpha \in all(p)$, where $\alpha$ is a structure then we say that
structure $\alpha$ \emph{matches exactly} pattern $p$. We say that $\alpha$ \emph{matches} $p$ if any of structure $all(p)$ is a sub-structure of $\alpha$.

In this thesis we only consider sequential pattern structures and use \emph{pattern} to mean \emph{sequential pattern structure}. We represent such patterns with regular expressions\insertref{regexp}. 

\tow{about regexps}

\emph{Pattern size} is the length of the pattern sequence.

\begin{exmp}
\R{.[AT]} is a pattern of size 2 and denotes a set 
\{ \R{AA}, \R{AT}, \R{CA}, \R{CT}, \R{GA}, \R{GT}, \R{TA}, \R{TT}\}; it matches \R{CCTC} and exactly matches \R{AT}.	
\end{exmp}

We denote the set of all pattern $p$ \emph{matches} in a dataset $D$ as $matches(p, D) = \{ match(p, \alpha) | \alpha \in D, matches(p, \alpha) \}$.

\section{Query}

We need to somehow understand where given pattern $p$ is located in a dataset $D$. This compound structure $q = <D, p, matches(p, D)>$ is called a \emph{query}.

\begin{exmp}
Let out dataset be $D = [ \text{\R{ACGT}}, \text{\R{TXCGA}} ]$ and our pattern be $p = \text{\R{C.}}$. The corresponding query is $<D, p, \{ [1,3], [2,4]\}>$, which means that the pattern $p$ ends in sequence 1 at position 3 and in sequence 2 at position 4.
\end{exmp}

\subsection{Query features}

When we talk about how "interesting" a pattern is, we are actually evaluating the query, since the pattern requires a context where it can be "interesting".

Queries can have different properties: length, number of matches in the dataset, pattern textual representation etc. Such properties can be represented by a function that take a query as an input and return the property. Formally a \emph{query feature} is a function $f: Query \mapsto Any$.

We also need to see how "interesting" one query is compared to the others. \emph{Query interestingness} is a function $f: Query \mapsto Value$ where the $Value$-s are well-ordered. This gives a measure to compare two different queries. Often we can represent such interestingness measures as a real number.

We should also be able to somehow specify criterias for query. \emph{Query filter} is a function $f: Query \mapsto Boolean$ and shows whether the query matches the criteria.

\begin{exmp}
Pattern occurences in a document is a interestingness measure. Whether query pattern is at least 3 tokens is a query filter.
\end{exmp}

\section{Pool}

\emph{Pool} is an abstract datatype for a collection of queries. The pool allows queries to be stored. The only operations that pool must provide is "push", for adding a query, and "pop", for getting a query.

\begin{exmp}
Stacks and queues both satisfy the pool requirement. We could also define a pool that stores the queries on the disk; also it could pack or reorder the queries for performance reasons.
\end{exmp}

\section{Pattern Discovery}

In this thesis \emph{pattern discovery} is a process of finding the most interesting subset, according to a query interestingness, of sequential patterns, that conform to some criteria, in a sequence dataset.

\begin{exmp}
Let our search problem be "Finding most common nucleotide patterns that are at least 3 nucleotides long from a shotgun sequencing output.", then \emph{most common} defines our interestingness measure. \emph{At least 3 nucleotides} is the pattern subset criteria. \emph{Sequencing output} is our dataset and \emph{nucleotides} define the token alphabet.
\end{exmp}
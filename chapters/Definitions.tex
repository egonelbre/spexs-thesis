\chapter{Definitions}

Pattern discovery is a research area aiming to discover unknown patterns
in a given set of data structures that are frequent and interesting according 
to some measure.

Since the discovery algorithms are highly dependent on the patterns, that are being searched, the algorithm must be minimal in the requirements on the dataset to be applicable as wide range of inputs as possible. This also means that the pattern structures are dependent on the initial data. \eg

\section{Sequence and Dataset}

We use $\Sigma$ to denote the set of tokens in the dataset, an \emph{alphabet}. 
The \emph{size} of the alphabet is $|\Sigma|$. \emph{Tokens} can be numbers, 
letters, words or sentences - any symbol.

Any sequence $S=a_1 a_2 ... a_n, \forall a_i \in \Sigma$ is called a \emph{sequence} 
over the token set $\Sigma$. If the length of the
string is $0$, it is called an empty sequence or $\epsilon$.

For example $ACGTGCCATC$ is a sequence where $\Sigma = \{A, C, G, T\}$.

A \emph{dataset} is a set of sequences. For example a document can be considered as a dataset, where the sentences are sequences and each word is a token in the alphabet. \eg

\section{Pattern}

\todo[inline]{use simpler definition for pattern}

A \emph{pattern} is a structure that defines a set of sequences. 
Pattern is usually denoted by a sequence over an extended alphabet. \eg

\tow{pattern size}

We denote the set of sequences that a pattern $p$ defines as $all(p)$.
If $\alpha \in all(p)$, where $\alpha$ is a sequence then we say that
sequence $\alpha$ \emph{matches exactly} pattern $p$. We say that $\alpha$
\emph{matches} $p$ if any of sequences $all(p)$ is a subsequence of $\alpha$.

The most commonly used pattern descriptions are regular expressions. For example $.[AT]$ denotes a set $\{AA, AT, CA, CT, GA, GT, TA, TT\}$ and this would match $CCTC$ and exactly match $AT$. A \emph{match} is a location set where the pattern in the sequence ends.

We denote the set of all pattern $p$ \emph{matches} in a dataset $D$ as $matches(p, D) = \{ match(p, \alpha) | \alpha \in D, matches(p, \alpha) \}$.

\section{Query}

\tow{motivation}

A \emph{query} is a compound structure $<D, p, matches(p, D)>$, 
where $D$ is the dataset and $p$ a pattern. \eg

\subsection{Query features}

\tow{motivation}

\emph{Query feature} is a function $f: Q \mapsto A$, 
where $Q$ is the query type. It gives information about 
the query such as the pattern representation, 
length, number of matches in the dataset.

\emph{Query interestingness} is a function $f: Q \mapsto Ord$ 
of the query whose results are well-ordered. This functions result allows to
say whether one query is more interesting than the other.
For convenience it is useful to represent that value in
$\Re$.

\emph{Query filter} is a function $f: Q \mapsto Bool$ of the query whose result
is a boolean.

\subsection{Pool}

\tow{proper definition}

Pool is an abstract datatype for a collection of queries. The pool allows queries to be put into it and taken from it, also we can ask whether the pool is empty or not.

It has no guarantees on how the queries are stored internally and in which order they are taken out. This gives an option to persist the queries if needed.

\eg

\tow{in practice}

\section{Pattern Discovery}

In this thesis \emph{pattern discovery} is a process of finding the most interesting subset of sequential patterns, according to some query interestingness, in a sequence dataset.

\tow{other possibilities}

For example "Finding most common nucleotide patterns that are at least 3 nucleotides long from a shotgun sequencing output." \emph{Most common} defines our interestingness measure. \emph{At least 3 nucleotides} is the pattern subset criteria. \emph{Sequencing output} is our dataset and \emph{nucleotides} define the token alphabet.
\chapter{Definitions}

Pattern discovery is a problem of finding interesting patterns in
some dataset. We discuss the algorithm in terms of DFAs and
show how this can help to discover patterns in sequences.

\subsection{Sequence}

We use $\Sigma$ to denote the set of tokens in the dataset, an \em{alphabet}. 
The \em{size} of the alphabet is $|\Sigma|$. \em{Tokens} can be numbers, 
letters, words or sentences - any symbol.

Any \em{sequence} $S=a_1 a_2 ... a_n, \forall i \in \Sigma$ is called a \em{sequence} 
over the token set $\Sigma$. If the length of the
string is $0$, it is called an empty string or $\epsilon$.

\subsection{Pattern}

A \em{pattern} is a structure that defines a set of structures $\Gamma$.
We denote the set that a pattern defines as $x(\Gamma)$.
If a structure $\alpha$ \em{matches} a pattern $\Gamma$, it means that
$\alpha \in x(\Gamma)$.

The most common form of such structures are regular expressions. 

\subsection{Finite automaton}

A \em{deterministic finite automaton} (henceforth \em{DFA}) is an 
finite state machine (henceforth \em{FSM}) that accepts or rejects
a sequence of tokens.

A \em{non-deterministic finite automaton} (henceforth \em{NFA}) is an 
FSM that accepts or rejects a sequence of tokens.

Patterns can be represented 

Both patterns and sequences can be represented by NFAs and hence
is a good abstraction for both. This also means that if our algorithm
works on NFAs it must work on any other dataset that is defined in 
terms of NFA.

\subsection{Dataset}

A \em{dataset} is a set of NFAs. In practice it may be more comfortable
to view it as a set of sequence because of it's simpler structure.

\subsection{Query}

A \em{query} is a compound structure that has information about a
pattern and it's matches in a dataset. The information about a
match is the ending states where a pattern matches a DFA.

On sequences this means the ending positions of the pattern 
in the sequence.

\subsection{Query interestingness}

\em{Query feature} is a function of the query. It gives 
information about the query such as the pattern representation, 
length, number of matches in the dataset.
\em{Query interestingness} is a function of the query whose
results are well-ordered. This functions result allows to
say whether one query is more interesting than the other.
For convenience it is useful to represent that value in
$R$.

\subsection{Pattern discovery problem}

\em{Pattern discovery problem} is a process of finding
most interesting querys by some interestingness measure.


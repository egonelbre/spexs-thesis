\chapter{Definitions}

Pattern discovery is a research area aiming to discover unknown patterns
in a given set of data structures that are frequent and interesting according 
to some measure.

Since the discovery algorithms are highly dependent on the
data structures, that are being searched, the algorithm must be minimal
in the requirements on the dataset to be applicable as wide range as possible.
This also means that the patterns found must be dependent on the initial data.


\section{Sequence and Dataset}

We use $\Sigma$ to denote the set of tokens in the dataset, an \emph{alphabet}. 
The \emph{size} of the alphabet is $|\Sigma|$. \emph{Tokens} can be numbers, 
letters, words or sentences - any symbol.

Any \emph{sequence} $S=a_1 a_2 ... a_n, \forall i \in \Sigma$ is called a \emph{sequence} 
over the token set $\Sigma$. If the length of the
string is $0$, it is called an empty sequence or $\epsilon$.

For example $ACGTGCCATC$ is a sequence where $\Sigma = \{A, C, G, T\}$.

A \emph{dataset} is a set of sequences. For example a document can be considered
as a dataset, where the sentences are sequences and each word is a token in the alphabet.

\section{Pattern}

A \emph{pattern} is a structure that defines a set of sequences. 
This pattern usually is denoted by a squence with an extended alphabet.

We denote the set of sequences that a pattern $p$ defines as $all(p)$.
If $\alpha \in all(p)$, where $\alpha$ is a sequence then we say that
sequence $\alpha$ \emph{matches exactly} pattern $p$. We say that $\alpha$
\emph{matches} $p$ if any of sequences $all(p)$ is a subsequence of $\alpha$.

The most commonly used pattern descriptions are regular expressions.
For example $.[AT]$ can denote a set $\{AA, AT, CA, CT, GA, GT, TA, TT\}$ and
this would match $CCTC$ and exactly match $AT$. A \emph{match} is a location
set where the pattern in the sequence ends.

We denote the set of all pattern $p$ \emph{matches} in a dataset 
$D$ as $matches(p, D) = \{ match(p, \alpha) | \alpha in D, matches(p, \alpha) \}$.

\section{Query}

A \emph{query} is a compound structure $<D, p, matches(p, D)>$, 
where $D$ is the dataset and $p$ a pattern.

\section{Query features}

\emph{Query feature} is a function of the query. It gives 
information about the query such as the pattern representation, 
length, number of matches in the dataset.

\emph{Query interestingness} is a function of the query whose
results are well-ordered. This functions result allows to
say whether one query is more interesting than the other.
For convenience it is useful to represent that value in
$\Re$.

\emph{Query filter} is a function of the query whose result
is a boolean.

\section{Pattern Discovery}

In this thesis \emph{pattern discovery} is a process of finding 
the most interesting patterns, according to some query interestingness,
in a dataset.

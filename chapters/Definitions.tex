\chapter{Definitions}

Pattern discovery is a problem of finding interesting patterns in
some dataset. We discuss the algorithm in terms of DFAs and
show how this can help to discover patterns in sequences.


\section{Sequence}

We use $\Sigma$ to denote the set of tokens in the dataset, an \emph{alphabet}. 
The \emph{size} of the alphabet is $|\Sigma|$. \emph{Tokens} can be numbers, 
letters, words or sentences - any symbol.

Any \emph{sequence} $S=a_1 a_2 ... a_n, \forall i \in \Sigma$ is called a \emph{sequence} 
over the token set $\Sigma$. If the length of the
string is $0$, it is called an empty sequence or $\epsilon$.

\section{Pattern}

A \emph{pattern} is a structure that defines a set of structures $\Gamma$. \todo{better definition}
We denote the set that a pattern defines as $x(\Gamma)$.
If a structure $\alpha$ \emph{matches} a pattern $\Gamma$, it means that
$\alpha \in x(\Gamma)$.

The most common form of such structures are regular expressions.

\section{Finite automaton}

A \emph{deterministic finite automaton} (henceforth \emph{DFA}) is an 
finite state machine (henceforth \emph{FSM}) that accepts or rejects
a sequence of tokens. \todo{define}

A \emph{non-deterministic finite automaton} (henceforth \emph{NFA}) is an 
FSM that accepts or rejects a sequence of tokens. \todo{define}

Both patterns and sequences can be represented by NFAs and hence
is a good abstraction for both. This also means that if our algorithm
works on NFAs it must work on any other dataset that is defined in 
terms of NFA.

\section{Dataset}

A \emph{dataset} is a set of NFAs. In practice it may be more comfortable
to view it as a set of sequence because of it's simpler structure.

\todo[inline]{example}

\section{Query}

A \emph{query} is a compound structure that has information about a
pattern and it's matches in a dataset. The information about a
match is the ending states where a pattern matches a DFA.

On sequences this means the ending positions of the pattern 
in the sequence.

\todo[inline]{example}

\section{Query features}

\emph{Query feature} is a function of the query. It gives 
information about the query such as the pattern representation, 
length, number of matches in the dataset.

\todo[inline]{example}

\emph{Query interestingness} is a function of the query whose
results are well-ordered. This functions result allows to
say whether one query is more interesting than the other.
For convenience it is useful to represent that value in
$\Re$.

\todo[inline]{example}

\emph{Query filter} is a function of the query whose result
is a boolean.

\todo[inline]{example}

\section{Pattern discovery problem}

\emph{Pattern discovery problem} is a process of finding
most interesting querys according to some 
interestingness measure, and use the pattern they reflect.
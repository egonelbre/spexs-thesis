\chapter{Introduction}
\label{c:introduction}

\section{Motivation and background}

One of the problems arising in dataset analysis is the discovery of interesting patterns. These patterns can show how the dataset is formed, how it repeats itself or they can be characteristic to some particular subset of the data.

For example a protein motif in a genomic sequence could predict disease. Patterns in medical diagnoses could show relations between diseases. Repeating pattern in source code could show how code could be minimized. Patterns in event logs could find causes for an error event.

Research in pattern discovery is mainly driven by biology, which means most of the discovery algorithms have been designed for genomic sequences in mind. The techniques are usually constrained to the genomic sequences, but these algorithms could be useful elsewhere. The algorithms probably could be useful in other fields as well. Searching new ways of using pattern discovery should be actively researched, progresses in any field can help other fields as well.

With genomic sequences there is another problem, the amount of data\cite{HowIsGenomeDoing}. The data collected are growing with increasing speed, which means we need to use more computational resources to analyse them. This means the pattern discovery algorithms should take advantage of multicore processors, highly parallel processors and clusters.

\subsection{Pattern Discovery}

Patterns can arise from different places: images, text, sounds, sequences, graphs, signals and more. The pattern representation can vary depending on the data itself and the parts that the pattern captures. Patterns in graphs can be sub-graphs with some additional info. Patterns represented in an image can be a weighted collection of different features.

Pattern discovery is trying to find \emph{a priori} unknown patterns in data that are interesting by some formal criteria. We could look for patterns that are frequent in data or have a "interesting" structure. A pattern that occurs more frequently than expected by chance may be considered interesting. Graph pattern having circular shape may also be considered interesting.

When choosing a pattern structure we should take into account the data, expected results and efficiency. Theoretically we could choose an abstract pattern structure that can represent all possible patterns, but practically this could be inefficient and be irrelevant to the problem. Finding image features from a visualization of sound would not probably produce anything meaningful.

When choosing a pattern rating method we shuld take into account that many patterns can occur by chance.  A pattern that specifies "something" is very frequent although it isn't very meaningful. There have been proposed many rating functions such as pattern occurances, ratio of pattern occurances between datasets, binomial and hypergeometric probability estimate and Z-score.

Although the idea of pattern discovery there are many parameters to choose from and is a difficult problem.

\subsection{Algorithm parallelization}

Taking an existing algorithm and making it parallel can be sometimes easier than writing an parallel algorithm from scratch. Existing algorithms may have already proven themselves in practice and have already good concepts. Algorithm parallelization can be divided into three subproblems:

\begin{enumerate}
	\item generalizing the algorithm,
	\item decomposing to independent tasks and
	\item reifying the generalized version with parallelization in mind.
\end{enumerate}

Generalizing the algorithm means loosening the order constraints and using minimal abstract data types for data storage. Here mathematical definitions and formulation of the problem is helpful. The less constraints there are the more freedom we have to change the implementation side.

Decomposing the algorithm means dividing it into independed tasks that could be ran in parallel. This also means trying to minimize the interaction and dependencies that "algorithm pieces" have.

Reifying the algorithm means finding suitable data structures for parallelization and mapping the independent tasks to different processes. The suitable structures and mapping to processes is dependent on the target architecture. For example data-structures involving vector operations work better on highly parallel processors.

Generalizing and decomposition steps can also make the algorithm simpler. Generalization makes algorithm more applicable to other fields, since there are less dependencies on the original problem.

\section{Contributions of this work}

We have derived a new parallel algorithm called SPEXS2 for discovering interesting patterns from a set of sequences. We describe SPEXS2 in a generic way and show how to possibilities of extending it further.

The practical and "ideal" versions of an algorithm can often diverge due to performance and implementation details, therefore we also explain problems and possible solutions with implementing such algorithm. We also have provided a concise implementation of the algorithm that captures the generic description more closely. Then we show some possible applications for the algorithm and analyse parallelization benefits.

\section{Structure of the thesis}

In this thesis we explore an algorithm for parallel pattern discovery. We choose an existing algorithm SPEXS\cite{spexs} and show how it can be parallelized.

In Chapter \ref{c:definitions} we introduce the terminology. In Chapter \ref{c:algorithms} we give an overview of already existing algorithms and discuss why SPEXS\cite{spexs} was chosen as a base for parallelization. We generalize the SPEXS algorithm in Chapter \ref{c:generalization} and reify it in Chapter \ref{c:parallelization}. We discuss an implementation of the parallelized algorithm in Chapter \ref{c:implementation} and in Chapter \ref{c:results} show its possible applications and performance characteristics. The conclusions are presented in Chapter \ref{c:conclusions}.

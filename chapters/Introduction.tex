\chapter{Introduction}

\section{Motivation and background}

Collecting new data has been increasing more rapidly than algorithms and
computer processing power. The average size of each dataset has also
been increasing. This suggests that the only way to keep up with
analysis is to parallelize algorithms.

One of main drivers of such large datasets is analysis
of genomic and proteomic sequences. Regularities in such data can 
give new insights into how these patterns form and how 
they are related to the other features of the data.

In this thesis we explore an algorithm for finding patterns and show how
generalizations can make it scalable and flexible, and simpler both in 
theory and implementation compared with non-abstract version.

\section{Structure of the thesis}

In the the Chapter 2 we give definitions for terminology that is used to
describe the alogrithm. In the Chapter 3 we investigate the original
SPEXS algorithm and TEIRESIAS algorithm. We generalize the SPEXS algorithm
to get a flexible algorithm in Chapter 4 and we partition the algorithm
to make it suitable for parallelization in Chapter 5. In Chapter 6 we
discuss the implementation of SPEXS. In Chapter 7 we show possible applications
for the algorithm.

\chapter{Introduction}

\tow{at least 3-4 pages}

\section{Motivation and background}

Collecting new data is being collected more rapidly \cite{HowIsGenomeDoing} . The average size of each dataset has also been increasing. This suggests that the only way to keep up with analysis is to parallelize algorithms.

One of drivers of such large datasets is analysis of genomic and proteomic sequences. \tow{todo}

\tow{a priori vs some prior knowledge}

\tow{broad examples from nat processing, code}

\tow{usual limitations, alphabet, pattern language...}

\section{Structure of the thesis}

In this thesis we explore an algorithm for finding patterns and show how
generalizations can make it scalable and flexible, and simpler both in 
theory and implementation compared with non-abstract version.

\tow{the basic premise of parallelization}

In the the Chapter 2 we give definitions for terminology that is used to
describe the alogrithm. In the Chapter 3 we investigate the original
SPEXS algorithm \cite{spexs} and other algorithms \insertref{other algorithms}. We generalize the SPEXS algorithm
to get a flexible algorithm in Chapter 4 and we partition the algorithm
to make it suitable for parallelization in Chapter 5. In Chapter 6 we
discuss the implementation of SPEXS. In Chapter 7 we show possible applications
for the algorithm.

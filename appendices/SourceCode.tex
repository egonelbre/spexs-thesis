\section{source}

The source code in \emph{src} has the following structure:

\dirtree{%
	.1 src/.
		.2 spexs \DTcomment{algorithm definition}.
			.3 extenders/ \DTcomment{query extenders}.
			.3 features/ \DTcomment{query feature calculators}.
			.3 filters/ \DTcomment{filter implementations}.
			.3 pool/ \DTcomment{different queue implementations}.
			.3 database.go \DTcomment{sequence dataset definition}.
			.3 query.go \DTcomment{query definition}.
			.3 spexs.go \DTcomment{algorithm implementation}.
		.2 spexs2 \DTcomment{command-line utility}.
			.3 conf.go \DTcomment{configuration reader}.
			.3 dataset.go \DTcomment{dataset reader}.
			.3 features.go \DTcomment{parses and creates feature functions}.
			.3 help.go \DTcomment{prints help for the program}.
			.3 printer.go \DTcomment{prints the final output}.
			.3 runtime.go \DTcomment{profiling and live-view setup}.
			.3 setup.go \DTcomment{prepares everything for algorithm}.
			.3 spexs2.go \DTcomment{main-entry point}.
}

There are also additional packages:

\dirtree{%
	.1 src/.
		.2 debugger/ \DTcomment{debugger for concurrent processes}.
		.2 stats/ \DTcomment{statistical functions}.
			.3 binom/ \DTcomment{binomial p-value calculation}.
			.3 hyper/ \DTcomment{hypergeometric p-value calculation}.
		.2 utils/ \DTcomment{additional utility functions}.
		.2 bit/ \DTcomment{functions for bitmanipulation}.
		.2 set/ \DTcomment{set implementations}.
			.3 hash/ \DTcomment{hash table with entry per value}.
			.3 bin/ \DTcomment{hash table with bitvectors}.
			.3 trie/ \DTcomment{2-level hashtable with bitvectors}.
}

For compilation there are two scripts \emph{make.bat} and \emph{make.sh} 
that build the program into bin directory.